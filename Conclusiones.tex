% Documento de Observaciones, Conclusiones y Experiencias del desarrollo
%Proyecto Cinema
% 24/11/2012
%     

\documentclass[letterpaper,11pt]{article}
\usepackage[spanish]{babel}
\usepackage[dvips]{epsfig}  
\usepackage[T1]{fontenc}
\usepackage[latin1]{inputenc}
\usepackage{tabularx}
\usepackage{url}
\usepackage{pslatex}
\usepackage{float}

     \topmargin -1 cm
     \textheight 23 cm
     \hoffset -1.5 cm
     
\pagestyle{empty}
\newcommand{\PreserveBackslash}[1]{\let\temp=\\#1\let\\=\temp}  
\let\PBS=\PreserveBackslash
%\renewcommand{\familydefault}{ccr}
 

\usepackage[dvips]{graphicx} % LaTeX
  
\begin{document}

%%%%%%% Titulo
\begin{tabular}{c >{\PBS\centering\vspace{-2,7cm}}p{11,5cm} c}
  & \rule{11,5cm}{0.7pt} \newline \newline
 \textbf{\textsc{{\Large Documento de Observaciones, Conclusiones y Experiencias del desarrollo} }}\newline
 \rule{11,5cm}{0.7pt} & \newline
 %\epsfig{file=mave.eps,width=70pt}
\end{tabular} 




\section{Observaciones}

\begin{tabular}{lll}

La aplicacion corre sin niguna problema en el emulador y en el dispositivo movil, pero la \\
aplicacion tiene algunas limitantes en cuanto al funcionamiento como por ejemplo:\\\\
\textbf{- Solo permite un pedido a la vez} \\
\textbf{- Al realizar el pedido lo guarda y finaliza sesion} \\\\
Cabe recalcar que estas limitaciones solo son debido al tiempo empleado en el desarrollo, \\
ya que teniendo un mayor tiempo y dedicacion completa se podra resolver estas limitantes.\\
\end{tabular}

%%%%% ESTUDIOS
\section{Conclusiones}

\begin{tabular}{lll}

 El proyecto de Cinema fue desarrollado en el programa ECLIPSE para celulares con \\
sistema operativo ANDROID y para guardar los datos se uso una base local y una base\\
en PHPMYADMIN \\


\end{tabular}

%%%%% CURSOS Y SEMINARIOS
\section{Experiencias}

\begin{tabular}{lll}

La experiencia final del proyecto fue muy gratificante ya que nos permitio conocer una nueva \\
plataforma de desarrollo que esta en pleno crecimiento en el mercado mundial en cuestion de \\
desarrollo de aplicaciones.\\\\
La experiencia en ciertas partes del codigo fue algo preocupante, debido a que como principiantes \\
en el desarrollo desconociamos como esta estructurado un proyecto ANDROID.\\\\
Ademas tuvimos problemas en hacer la conexion a una base de datos externa e interna, las que\\
finalmente supimos hacerlas\\ \\

\end{tabular}

%%%%% ENLACES DE AYUDA
\section{Enlaces de ayuda en el desarrollo}

\begin{tabular}{lll}

- Como desarrollar aplicaciones para Android.(n.f.). Recuperada Noviembre 6, 2012, de \\
http://blog.openalfa.com/como-desarrollar-aplicaciones-para-android/.\\\\
- Lanzar un segundo Activity.(n.f.). Recuperada Noviembre 6, 2012, de \\
http://www.javaya.com.ar/androidya/detalleconcepto.php?codigo=140 inicio=.\\\\
- Lanzar un segundo Activity y pasar parametros.(n.f.). Recuperada Noviembre 6, 2012, de \\
http://www.javaya.com.ar/androidya/detalleconcepto.php?codigo=141inicio=.\\\\
- Almacenamiento en una base de datos SQLite.(n.f.). Recuperada Noviembre 6, 2012, de \\
http://www.javaya.com.ar/androidya/detalleconcepto.php?codigo=145inicio=.\\\\
- Usando Google Maps con Android.(Octubre 7, 2011.). Recuperada Noviembre 6, 2012, de \\
http://androideity.com/2011/10/07/usando-google-maps-con-android/.\\\\
- Login en Android usando PHP y MySQL.(Julio 5, 2012.). Recuperada Noviembre 6, 2012, de \\
http://androideity.com/2012/07/05/login-en-android-usando-php-y-mysql/.\\\\
- Testear aplicaciones Android en tu telefono.(Octubre 3, 2011.). Recuperada Noviembre 6, 2012, de \\
http://androideity.com/2011/10/03/testear-aplicaciones-android-en-tu-telefono/.\\\\
\end{tabular}


\end{document}
